\documentclass{article}
\usepackage{fullpage}
\begin{document}

\title{Manual of Disambiguation}
\author{Ye (Edward) Sun}
\date{August, 2011}
\maketitle

\section{Overview}

The source code is an implementation of authorship 
disambiguation algorithm described by Vetle
Torvik et al 2009. While it is originally designed 
for the disambiguation of inventors in the United
States Patent and Trademark Office (USPTO) database, 
minor modifications in the source file can
provide solutions for problems of the similar kind.

The algorithm uses Bayesian theorem to calculate the 
probability of match between two records based
on the comparison between the two records in significant 
attributes such as names, geographical location and companies, 
etc. The algorithm includes three major steps: 1. Blocking,
 2. Training, and 3. Disambiguation. 

The details of each step are described in the following sections.

The source code is written in C++ and uses many features 
of C++, such as polymorphism, template, STL library, etc. 
Therefore, a fairly good understanding of C++ is expected 
to fully understand the code. The code also outsources 
IBM CPLEX for quadratic programming. Import and export of the
disambiguation result may need some APIs of SQL/SQLite. 
The code is compiled in Linux by GNU gcc/g++/make.


\section{Work Flow}

\subsection{Schematic}

This is a schematic of the whole disambiguation process. 
Each array represents one step, with the
square box of identical color representing the source 
code files that implements the step, and the round
box of identical color representing the necessary 
definitions of relevant classes and/or functions.


The disambiguation is an iterative process: 
the result of previous disambiguation serves as the input of
the next round disambiguation. For each round of disambiguation, 
there are three major steps: blocking, training and disambiguation.

\subsection{Blocking}

Blocking of records is the first step of disambiguation. 
The whole point of blocking is to avoid most
unnecessary comparisons without much accuracy loss. Records 
of similar attributes are blocked
together since they are likely to be a match. Because the 
aim of disambiguation is individual inventors,
the most significant attributes – names – are 
usually adopted as the blocking rules. It is critical to
remember that ONLY records within the same blocks are 
compared and thus disambiguated.


Iterative blocking in a permissive way is applied in the 
disambiguation, because it is assumed that after each round 
of disambiguation, blocks are well disambiguated and thus 
similar and/or more stringent blockings are meaningless. 
For example, the first round of blocking is “exact 
first names + exact last names”, the second can 
be “first 5 characters of first names + first 8 characters 
of last names”, and the third round can be “first 
3 characters of first names + first 5 characters of last names”.


\subsection{Training}

The point of training is to obtain the ratios database for 
similarity profiles based on training sets.

\subsubsection{Similarity Profiles}

Similarity profiles are defined as multi-dimensional variables 
that describe the similarity of two records. Each record has 
many attributes, such as first name, last name, assignee, street, 
city, country, etc. Similarity profiles are selected attributes 
that are believed to be influential in pairwise comparisons; thus, 
they are decided by users and can be different in each round of 
disambiguation. For example, similarity profiles in the first round 
of disambiguation can be “first name + last name + assignee + 
city”, while those in the second round can be “ last name + 
assignee + street ”. Each attribute in a similarity profile is 
given a score to evaluate the similarity of the attribute from the
comparison of the two records. The score is discrete and is usually 
non-negative integers. Scoring of each attribute is totally 
user-defined, and, conceptually, has to be fully understood by users.
Similarity profiles refer to the multi-dimensional integers obtained 
from the comparison.



\subsubsection{Training sets}

Training sets contain pairs of records that are believed 
with high confidence to be either match or
non-match. In the USPTO patent inventor disambiguation, 
there are four training sets:

\begin{enumerate}

\item match pairs to train personal information based 
on patent information: this set is created by selecting 
record pairs belonging to the same blocks and having at 
least two common coauthors.

\item non-match pairs to train personal information based 
on patent information: this set is created by
selecting record pairs that share the same patent id but 
different author sequence of that patent.

\item match pairs to train patent information based on 
personal information: this set is created by
selecting record pairs that share exact rare names.

\item non-match pairs to train patent information based on 
personal information: this set is created by
selecting record pairs whose names are both rare and different.

\end{enumerate}


\subsubsection{Ratios database}


The ratio value of a given similarity profile, or R-value, 
is defined as the frequency of the similarity profile in the 
match set divided by the frequency of the same similarity 
profile in the non-match set. Ratios database, therefore, 
is the mapping of all possible similarity profiles to 
their corresponding r-values. In the patent disambiguation, 
personal information is assumed to be independent from patent
information, so r-value of a similarity profile is the 
multiple of the r-value of the personal information
part of the similarity profile and that of the patent 
information part of it. By using the four training sets
described above, the ratios database can be obtained. Usually 
the ratios database is implemented as a binary search tree. 
Since it is likely that some similarity profiles are not 
observed in match or non-match sets, interpolation and 
extrapolation of those profiles become necessary. Moreover, the
similarity profiles are assumed to be monotonic. Quadratic 
programming is adopted to accomplish the monotonicity 
enforcement, the interpolation and the extrapolation.


\subsection{Disambiguation}


\subsubsection{Clusters}

The smallest unit of disambiguation is a cluster, 
which contains one or more records. A cluster is
interpreted as a unique inventor. The ultimate 
goal of disambiguation is to finalize clusters 
and their components.


\subsubsection{Probability of Match}

When two records are compared to find out the probability 
of match, their similarity profiles are calculated first 
by comparing relevant attributes. Then the corresponding 
r-value is looked up from the ratios database. Finally, the 
Bayesian theorem is applied to get the final probability 
of match, with the help of a priori probability which 
is block-dependent.



\subsubsection{Cluster-based Disambiguation}

During disambiguation, clusters belonging to the same 
blocks are compared. No inter-block cluster comparisons 
would happen. When two clusters are compared, their 
component records are compared exhaustively to find the 
interactive force between the two clusters. The sum of 
the interactive probabilities is compared to internal 
probabilities of both clusters. If the comparison passes a
predetermined threshold, the two clusters are determined 
as of a same inventor and are merged together. The process 
continues independently within each block until no merge 
can happen. Once all blocks are disambiguated, the 
disambiguation process of the current round ends.


\section{Configuration}


\subsection{Prerequisites}


\begin{itemize}

\item A high-end workstation (usually 8 CPUs and 
24G RAM). 24G RAM is probably the minimum memory 
requirement. Since multi-threading is supported, 
more CPUs generally reduce time cost.

\item Linux ( Ubuntu, Fedora or other compatibles )

\item GCC/G++ (4.0 or higher)

\item IBM CPLEX

\item SQLite3 (3.6 or higher) (This is not mandatory 
but necessary if one needs to out put the result to
a sqlite3 database).

\end{itemize}


\section{Configuration and Compilation of Executables}

\subsection{Configure IBM CPLEX}

IBM CPLEX package for academia is usually free for download 
and use. Trial versions will not work as limitations apply. 
Follow the instructions to install. Pay attention to the 
architecture (32/64 bit).


\subsubsection{Configure SQLite3}

SQLite3 is a public domain, light weight SQL database engine. 
Download from website and install.


\subsubsection{Configure Makefile}

Edit the file “makefile” in the source code package. 
Things include

\begin{itemize}

\item The path of IBM CPLEX library and header files 
(pay attention to architecture.)

\item The path of SQLite3 library and header files

\item System libraries and header files (math library, 
pthread library, etc).

\item Optimization flags (optional).

\end{itemize}



\subsubsection{Compile}

In the command line, in the directory where the 
file “makefile” is, type “make”. 
Source codes will compile automatically. A successful 
compilation will generate several executables, namely
“exedisambig” and “txt2sqlite3”. 
The former is the main disambiguation program, and the 
latter is the executable that dumps the disambiguation 
result into sqlite3 databases. Errors in compilation are 
usually caused by incorrect configurations in 
the “makefile” file. If necessary, type 
“make clean” to remove all the newly built 
object files, and “make” to rebuild.


\subsection{Configuration of Disambiguation}

The disambiguation is configured by two parts: 
the disambiguation engine and the blocking rules.


\subsubsection{Engine Configuration}

The configuration of the disambiguation engine includes the following items:

 Working directory: the directory where all the intermediary 
and final files will be saved.

 Original CSV File: the source database of disambiguation in 
text format, which is usually exported from a SQL database. 
The first line of the file consists of names of each
attribute/column, and the rest lines are data.

 Number of threads: the number of maximum threads to 
allow multi-threading of the program.

Usually it is set to be the number of CPUs in the computer.

 Generate stable training sets: match and non-match training sets from rare names are usually
very stable. Therefore, if those training sets already exist, this switch can be set to “false”.
However, since the creation of the stable training sets generally does not take too much time,
it is usually recommended to be “true”.
 Use available ratios database: this switch controls whether or not training should take place.
Since training is usually somewhat time-costly (~4 hours), this switch can be set to “true” if
training is unnecessary, such as when debugging. But generally this option should be set to
“false”.
 Thresholds: the thresholds series to determine whether or not two records are a match should
be set in a permissive way, such as “0.99, 0.98, 0.95”. The reason to do this is to allow most
similar records conglomerate first, which can help improve the accuracy.
 Necessary attributes: This option selects those necessary attributes to load into the memory
from the original CSV file, because not all the data in the original CSV file need to be loaded.
These necessary attributes should be exactly the same as those appearing in the first line of
the CSV file.
 Adjust prior by frequency: This option controls the evaluation of the priori probability for
each block. It is an option that should always set to be “true” unless modification of the code
occurs after thorough understanding of the code.
 Debug mode: this switch indicates whether the program is for debugging or not. Set to “false”
if a normal disambiguation is desired. In debug mode, the program will look for the debug
configuration file “debug\_block\_x.txt” in the working directory, where x is the current round
of disambiguation. The debug configuration file contains all the specified blocks that are of
interest, and the program only disambiguates those blocks.
 Number of training pairs: the maximum of pairs of records obtained for training. Usually set
to 10 million.
 Starting round: this number specifies the round of disambiguation to start with.
 Starting file: this string specifies the file of previous disambiguation result from which the
disambiguation starts. If the starting round is 1, the starting file just needs to be a valid
location in the file system, because it will be automatically created or overwritten.
 Postprocess after each round: the switch determines whether or not post-processing should be
applied after each round of disambiguation. It can be either “true” or “false”, although “true”
is recommended.


\subsubsection{Blocking Configuration}


The blocking configuration file determines the blocking 
rules and the similarity profile components
for each round of disambiguation. It is in the format of:

[ Round X ]

Attribute Name 1: String Manipulation parameters
…
Attribute Name M: String Manipulation parameters
Active Similarity Attributes: attribute names of similarity profile components.

[ Round Y ]

Attribute Name 1: String Manipulation parameters
…
Attribute Name N: String Manipulation parameters
Active Similarity Attributes: attribute names of similarity profile components.
In the current engine, a typical example is:

[ Round 5 ]

Firstname: 1 : 0 , 3 , true
Middlename: 1 : 0 , 0 , false
Lastname: 0 : 0, 5, true
ACTIVE SIMILARITY ATTRIBUTES: Firstname, Middlename, Lastname, Latitude, Coauthor,
Assignee
This means that the blocking identification for round 5 is created from three attributes:
 First name: the raw data is read from the index 1 position of the first name attribute object ( see
more details in source codes), and is truncated from the index 0 position of the string for a
maximum length of 3 characters in a forward direction ( left to right ).
 Middle name: the raw data is read from the index 1 position of the middle name attribute object
( see more details in source codes), and is truncated from the index 0 position of the string for a
maximum length of 0 characters in a backward direction ( right to left ).
 Last name: the raw data is read from the index 0 position of the last name attribute object ( see
more details in source codes), and is truncated from the index 0 position of the string for a
maximum length of 5 characters in a forward direction ( left to right ).
And the components of similarity profiles are: first name, middle name, last name, latitude
(geographical location), coauthor, assignee (company).


\subsection{Export Disambiguation Results}

The results of disambiguation can be exported to a 
SQLite database by running the \texttt{txt2sqlite3}
executable file.



\section{Implementation Details}


\subsection{Data Structure}


The disambiguation engine is designed in an object-oriented 
way. The key concepts in the whole engine include:


 Attributes: conceptualized as the information in each column in the database. The attribute
class has a complicated hierarchy and thus requires much more understanding. As for the
implementation, each CONCRETE attribute class has multiple pointers to data strings. See
more details in the source code.
 Records: a vector of attributes representing a full record in the database. More strictly, each
record contains a vector of pointers to attributes.
 Clusters: conceptualized as a unique inventor having multiple records. Each cluster contains a
list of pointers to records.
The following picture shows the data structure of the disambiguation.


\subsection{Customization of Attributes}


The attribute class has several layers of inheritance, 
each of which implements some more functionality and thus 
becomes more concrete. At the  very low level of the 
hierarchy, there are several modes to choose. Ideally one 
only needs to pick up a mode to inherit if a concrete class 
is supposed to be defined. It is also the class creator’s 
responsibility to know and to define (or override) the 
following data members and member functions for a 
given concrete class:

 Class name: the name of the concrete class.
 Class name specifier string: the name specifier of the concrete class, which can be used to
retrieve data from the original data file.
 Class group: the training group of the class, such as personal side, patent side, or none.
 Name specifiers of interactive classes: the specifiers of the classes that is necessary for the
current class in order for a viable comparison definition.
 Number of interactive classes: the number of interactive classes.
 Maximum value: the maximum score of the comparison function of the class.
 Compare: the comparison function of the class.
 Other functions if necessary.
Here is a typical example of the customization of the Firstname attribute and the Latitude attribute.
For Firstname:
 Class name: cFirstname. This class has no interaction with other classes and thus simply
inherits from the mode “cAttribute\_Single\_Mode<cFirstname>”.
 Class name specifier string: “Firstname”. This is exactly the same as the one in the original
database.
 Class group: “Personal”. First name is in the personal information side.
 Name specifiers of interactive classes: {} (empty).
 Number of Interactive classes: 0
 Maximum value: 4. The rating is user-defined and has to be consistent with the comparison
function.
 Compare: Override the parent class definition.
 Other functions:
 split\_string: defines how the data is extracted from the original database and stored in the
class object.
 exact\_compare: defines how the “exact” comparison function works.
For Latitude:
 Class name: cLatitude. This class has interaction with several other attributes, so it inherits
from “cAttribute\_Interactive\_Mode<cLatitude, cLatitude\_Data>”, where cLatitude\_Data
contains the real data and inherits from “cAttribute\_Singe\_Mode<cLatitude\_Data>”.
 Class name specifier string: “Latitude”.
 Class group: “Patent”
 Name specifiers of interactive classes: {“Longitude”, “City”, “Country”}
 Number of interactive classes: 3
 Maximum value: 5.
 Compare: Overridden.

\subsection{Customization of Cluster Comparisons}

Comparison between two clusters, a critical part of 
disambiguation, includes several configurable
steps that can be very influential to the results.
When two clusters are compared:


 The two representatives of the clusters are compared to get a post-probability. If the
probability does not pass a minimum screening threshold, the clusters are judged to be of
different inventors and should not merge. The comparison of the clusters stops here.
 If the probability passes the minimum threshold, then:
o Another threshold, dependent on the features of the clusters, is calculated and will be
used as a caliber in the further disambiguation of the two clusters. The threshold
depends on the cohesions of the two clusters, the career gap between the clusters, the
number of common locations between the clusters, etc.
o A priority queue of a given size is created. The size of the queue depends on the size of
both clusters. The larger the clusters, the larger the queue size.
o Exhaustive comparisons between members of the two clusters are performed, and the
results are fed into the priority queue in the following manner.
 If the priority queue is not full, the new result fills into the queue.
 If the queue is full, the new result is compared with the lowest value in the
queue.
 If the new result is greater than the lowest value, and the lowest
value is less than the caliber threshold, then the new result simply
replaces the lowest value in the priority queue.
 If the new results is less than the lowest value but greater than the
caliber threshold, the new result is added to the priority queue.
 If the new result is less than the lowest value and also less than the
caliber threshold, discard the new result.
o The average value of the priority queue is then calculated after the exhaustive
comparisons finish. If the average is greater than the caliber threshold, the two
clusters are labeled as of a same inventor and will be followed by a merge of the
clusters; otherwise, the comparison of the clusters stops here.
 If the clusters are labeled as a merge, the two clusters will merge into a new bigger cluster,
with the previously calculated average value of the priority queue as its new cohesion value
(or with-in-cluster density). A new representative of the new cluster will be decided later.


Therefore, the customization of the cluster comparisons can be in:

\begin{itemize}

\item The screening threshold

\item The caliber threshold

\item The size of the priority queue

\item The way to calculate the new cohesion value (the with-in-cluster density)

\end{itemize}


\end{document}
